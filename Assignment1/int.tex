%iffalse
\let\negmedspace\undefined
\let\negthickspace\undefined
\documentclass[journal,12pt,twocolumn]{IEEEtran}
\usepackage{cite}
\usepackage{amsmath,amssymb,amsfonts,amsthm}
\usepackage{algorithmic}
\usepackage{graphicx}
\usepackage{textcomp}
\usepackage{xcolor}
\usepackage{txfonts}
\usepackage{listings}
\usepackage{enumitem}
\usepackage{mathtools}
\usepackage{gensymb}
\usepackage{comment}
\usepackage[breaklinks=true]{hyperref}
\usepackage{tkz-euclide} 
\usepackage{listings}
\usepackage{gvv}                                        
%\def\inputGnumericTable{}                                 
\usepackage[latin1]{inputenc}                                
\usepackage{color}                                            
\usepackage{array}                                            
\usepackage{longtable}                                       
\usepackage{calc}                                             
\usepackage{multirow}                                         
\usepackage{hhline}                                           
\usepackage{ifthen}                                           
\usepackage{lscape}
\usepackage{tabularx}
\usepackage{array}
\usepackage{float}


\newtheorem{theorem}{Theorem}[section]
\newtheorem{problem}{Problem}
\newtheorem{proposition}{Proposition}[section]
\newtheorem{lemma}{Lemma}[section]
\newtheorem{corollary}[theorem]{Corollary}
\newtheorem{example}{Example}[section]
\newtheorem{definition}[problem]{Definition}
\newcommand{\BEQA}{\begin{eqnarray}}
\newcommand{\EEQA}{\end{eqnarray}}
\newcommand{\define}{\stackrel{\triangle}{=}}
\theoremstyle{remark}
\newtheorem{rem}{Remark}

% Marks the beginning of the document
\begin{document}
\bibliographystyle{IEEEtran}
\vspace{3cm}

\title{Subjective questions}
\author{AI24BTECH11018 - Sreya}
\maketitle
\newpage
\bigskip
\begin{enumerate}

    

\renewcommand{\thefigure}{\theenumi}
\renewcommand{\thetable}{\theenumi}
\item[21.] Sketch the curves and identify the region bounded by $x=\frac{1}{2}$,$x=2$,$y=\ln{x}$ and $y=2^x$. Find the area of region.

\hfill{\brak{1991-4 Marks}}

\item[22.] if $f$ is a continous function with $\int_{0}^{x}f\brak{t}dt\rightarrow\infty$ as $\abs{x}\rightarrow\infty$, then show that every line $y=mx$\\

\begin{tikzpicture}[scale=1.5]
    % Draw the axes
    \draw[->] (0,0) -- (4,0) node[right] {$X$};
    \draw[->] (0,-2) -- (0,2) node[above] {$Y$};
    
    % Draw the parabola
    \draw[thick] plot[domain=-1.41:1.41] ({2-(\x)^2}, \x);
    
    % Points
    \fill (0,{sqrt(2)}) circle (1pt) node[left] {$(0,\sqrt{2})$};
    \fill (0,-{sqrt(2)}) circle (1pt) node[left] {$(0,-\sqrt{2})$};
    \fill ({2}, 0) circle (1pt) node[below right] {$(X_p,0)$};
    \node[below left] at (0,0) {O};
    
    % Label points A and B
    \node at (0,1.7) {A};
    \node at (0,-1.7) {B};

\end{tikzpicture}
intersects the curve $y^2+\int_{0}^{x}f\brak{t}dt=2!$

\hfill{\brak{1991-4 Marks}}\\

\item[23.] Evaluate $\int_{0}^{\pi}\frac{x\sin 2xs\sin \brak{\frac{\pi}{2}\cos x}}{2x-\pi}$

\hfill{\brak{1991-4 Marks}}

\item[24.]  Sketch the region bounded by the curves\\ $y=x^2$ and
$y=\frac{2}{1+x^2}$. Find the area.


\hfill{\brak{1992-4 Marks}}

\item[25.]  Determine a positive integer $n\leq 5$, such that $\int_e^x\brak{x-1}^n=16-6e$

\hfill{(1991-4 Marks)}\\

\item[26.]  Evaluate $\int_{2}^{3}\frac{2x^5+x^4-2x^3+2x^4+1}{\brak{x^2+1}\brak{x^4-1}}dx$

\hfill{\brak{1993 - 5 Marks}}\\

\item[27.]  Show that $\int_{0}^{n\pi+v}\abs{\sin x}dx=2n+1-\cos v$ where n is a positive integer and $0 \leq v < \pi$

\hfill{\brak{1994-4 Marks}}\\

\item[28.]  In what ratio does the x-axis divide the area of the region bounded by the parabolas $y=4x-x^2$ and $y=x^2-x$?

\hfill{\brak{1994-5 Marks}}\\

\item[29.]  $I_m=\int_{0}^{\pi}\frac{1-\cos mx}{1-\cos x}dx$. Use mathematical induction to prove that $I_m=m\pi$, $m=0,1,2,3,......$

\hfill{\brak{1995-5 marks}}\\

\item[30.]  Evaluate the definite integral:
$\int\limits_{\frac{-1}{\sqrt{3}}}^{\frac{1}{\sqrt{3}}}\brak{\frac{x^4}{1-x^4}}
cos^{-1}\brak{\frac{2x}{1+x^2}}$

\hfill{\brak{1995-5Marks}}\\

\item[31.]  Consider a square with vertices at \brak{1,1},\brak{-1,1},\brak{-1,-1} and \brak{1,-1}. Let s be the region consisting of all points inside  the square which are nearer to the origin than to any edge. Sketch the region S and find its area.

\hfill{\brak{1995-5 Marks}}\\

\item[32.]  Let $A_n$ be the area bounded by the curve\\ $y=\brak{tanx}^n$ and the lines $l=0$,$y=0$ and $x=\frac{\pi}{4}$. Prove that for $n>2$, $A_n+A_n-2=\frac{1}{n-2}$ and deduce\\ $\frac{1}{2n+2}<A_n<\frac{1}{2n-2}$.


\hfill{\brak{1996-3 Marks}}\\

\item[33.]  Determine the value of $\int_{-\pi}^{\pi}\frac{2x\brak{1+\sin x}}{1+\cos ^2x}dx$.


\hfill{\brak{1997-5 Marks}}\\

\item[34.]  Let $f\brak{x}$= Maximum {$x^2,\brak{1-x}^2,2x\brak{1-x}$}, Where $0 \leq x \leq 1$. Determine the area of the region bounded by the curves $y=f\brak{x}$,x-axis,$x=0$ and $x=1$.
\hfill{\brak{1997-5 Marks}}\\
\item[35.]  Prove that $\int_{0}^{1}\tan ^{-1}\brak{\frac{1}{1-x+x^2}}dx$=2$\int_{0}^{1} tan^{-1}xdx$. Hence or otherwise, evaluate the integral $\int_{0}^{1}\tan ^{-1}\brak{1-x+x^2}dx.$\\
\hfill{\brak{1998-8 Marks}}

\end{enumerate}


\end{document}
