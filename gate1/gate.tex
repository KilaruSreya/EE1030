\let\negmedspace\undefined
\let\negthickspace\undefined
\documentclass[journal]{IEEEtran}
\usepackage[a5paper, margin=10mm, onecolumn]{geometry}
%\usepackage{lmodern} % Ensure lmodern is loaded for pdflatex
\usepackage{tfrupee} % Include tfrupee package

\setlength{\headheight}{1cm} % Set the height of the header box
\setlength{\headsep}{0mm}  % Set the distance between the header box and the top of the text

\usepackage{gvv-book}
\usepackage{gvv}
\usepackage{cite}
\usepackage{amsmath,amssymb,amsfonts,amsthm}
\usepackage{algorithmic}
\usepackage{graphicx}
\usepackage{textcomp}
\usepackage{xcolor}
\usepackage{txfonts}
\usepackage{listings}
\usepackage{enumitem}
\usepackage{mathtools}
\usepackage{gensymb}
\usepackage{comment}
\usepackage[breaklinks=true]{hyperref}
\usepackage{tkz-euclide} 
\usepackage{listings}
% \usepackage{gvv}                                        
\def\inputGnumericTable{}                                 
\usepackage[latin1]{inputenc}                                
\usepackage{color}                                            
\usepackage{array}                                            
\usepackage{longtable}                                       
\usepackage{calc}                                             
\usepackage{multirow}                                         
\usepackage{hhline}                                           
\usepackage{ifthen}                                           
\usepackage{lscape}
\usepackage{tikz}
\usetikzlibrary{patterns}
\usepackage{circuitikz}
\begin{document}

\bibliographystyle{IEEEtran}
\vspace{3cm}

\title{2021-MA- 14-26}
\author{AI24BTECH11018 - Sreya}

% \maketitle
% \newpage
% \bigskip
{\let\newpage\relax\maketitle}

\renewcommand{\thefigure}{\theenumi}
\renewcommand{\thetable}{\theenumi}
\setlength{\intextsep}{10pt} % Space between text and floats

\begin{enumerate}
\setcounter{enumi}{13}
\item Consider the following topoligies on the set $\mathbb{R}$ of all real numbers:\\
$T_1$={$U \subset \mathbb{R} :0 \notin U$ or $U=\mathbb{R}$},\\
$T_2$={$U \subset \mathbb{R} :0 \in U$ or $U=\phi$ }\\
$T_3=T_1 \cap T_2$\\
Then the closure of the set ${1}$ in $\brak{\mathbb{R},T_3}$
\begin{enumerate}
    \item ${1}$
    \item $0,1$
    \item ${\mathbb{R}}$
    \item $\mathbb{R} \texttt{\textbackslash} {0}$
\end{enumerate}
\item Let $f:{\mathbb{R}}^2\rightarrow \mathbb{R}$ be a differentiable.Let $D_uf\brak{0,0}$ and $D_vf\brak{0,0}$ be the directinal derivatives at of $f$ at $\brak{0,0}$ in the direction of unit vectors $u=\brak{\frac{1}{\sqrt{5}},\frac{2}{\sqrt{5}}}$ and $v=\brak{\frac{1}{\sqrt{2}},\frac{-1}{\sqrt{2}}}$. respectvely. if $D_uf\brak{0,0}=\sqrt{5}$ and $D_vf\brak{0,0}=\sqrt{2}$ then $\frac{\partial f}{\partial x}\brak{0,0}$+$\frac{\partial f}{\partial y}\brak{0,0}=$
\item let $r$ denote the boundary of the square region $R$ with vertices $\brak{0,0}$,$\brak{2,0}$,$\brak{2,2}$ and $\brak{0,2}$ oriented in the counter-clockwise direction. then \begin{equation}
\oint_r \brak{1-y^2}dx+x dy=
\end{equation}
\item The number of $5-sylow$ subgroups in symmetric groups $S_5$ of degree $5$ is 
\item let $I$ be the generated by $x^2+x+1$ in the polynomial ring $R=\mathbb{Z} \sbrak x$,where $\mathbb{Z}_3$ denotes the ring of integers modulo $3$. Then the number of units in the quotient ring $\frac{R}{I}$ is
\item Let $T:{\mathbb{R}}^3 \rightarrow {\mathbb{R}}^3$ bea linear transforation such that\\
$T\brak{\begin{array}{c}
1 \\
1 \\
1
\end{array}}=\brak{\begin{array}{c}
1 \\
-1 \\
1
\end{array}}$, $T^2\brak{\begin{array}{c}
1 \\
1 \\
1
\end{array}}=\brak{\begin{array}{c}
1 \\
1 \\
1
\end{array}}$,$T^2\brak{\begin{array}{c}
1 \\
1 \\
2
\end{array}}=\brak{\begin{array}{c}
1 \\
1 \\
2
\end{array}}$
\item Let $y\brak{x}$ be the solution of the following intial value problem 
\begin{equation}
    x^2{\frac{d^2{y}}{d{x}^2}}-4x\frac{dy}{dx}+6y=0,    x\textgreater 0\end{equation}
    \begin{equation}
    y\brak{2}=0,\frac{dy}{dx}\brak{2}=4.
\end{equation}
Then $y\brak{4}=$
\item Let 
\begin{equation}
    f\brak{x}=x^4+2x^3-11x^2-12x+36 for x\in \mathbb{R}
\end{equation}
The order of convergence of the newton-raphson method
\begin{equation}
    x_n+1=x_n-\frac{f\brak{x_n}}{f \prime \brak{x_n}} , n\geq 0,
\end{equation}
with $x_0=2.1$ for finding the root $\alpha =2$ the equaton $f\brak{x}=0$ is 
\item if the polynomial 
\begin{equation}
    p\brak{x}=\alpha + \beta \brak{x+2}+\gamma \brak{x+2}\brak{x+1}+\delta \brak{x+2}\brak{x+1}x
\end{equation}
interpolates the data\\ \\$\begin{tabular}{|c|c|c|c|c|c|}

\hline
x    & -2   & -1   & 0   & 1   & 2   \\
\hline
f\brak{x}    & 2   & -1   & 8   & 5   & -34  \\
\hline
\end{tabular}$\\\\
then $\alpha + \beta + \gamma + \delta$
\item Consider the linear programming problem $P$:
\begin{equation}
Maximise  2x_1+3x_2\end{equation}
subject to
\begin{equation}
    2x_1+x_2 \leq 6,
\end{equation}
\begin{equation}
    -x_1+x_2\leq 1, 
\end{equation}
\begin{equation}
    x_1+x_2\leq 3,
\end{equation}
\begin{equation}
    x_1\geq 0, and  x_2\geq 0
\end{equation}
Then the optional of the dual of $P$ is equal to 
\item Consider the linear programming problem $P$:
\begin{equation}
Maximise  2x_1+3x_2\end{equation}
subject to
\begin{equation}
    2x_1+3x_2+s_1=12,
\end{equation}
\begin{equation}
    -x_1+x_2+s_1= 1, 
\end{equation}
\begin{equation}
    x_1+2x_2+s_3=3,
\end{equation}
\begin{equation}
    x_1\geq 0,x_2\geq 0, s_1\geq 0,s_2\geq 0, and s_3\geq 0.
\end{equation}
if $\brak{\begin{array}{c}
x_1 \\
s_1 \\
s_2 \\
s_3
\end{array}}$ is a basic feasable solution of $p$ , then $x_1+s_1+s_2+s_2+s_4$
\item Let $H$ be a complex Hilbert space. Let $u,v\in H$ be a such that $\langle u, v \rangle=2$. Then 
\begin{equation}
    \frac{1}{2\pi}\int_{0}^{2\pi} \abs {\abs {u+e^{it}v}}^2 e^{it}dt=
\end{equation}
\item Let $\mathbb{Z}$ denote the ring of integers. consider the ring\\
$R={a+b\sqrt{-17}: a,b \in \mathbb{R}}$ of the feild $\mathbb{C}$ of complex numbers \\
Consider the following statements:\\
$P:2+\sqrt{-17}$ is an irreducible element.\\
$Q:2+\sqrt{-17}$ is a prime element.\\
Then
\begin{enumerate}
    \item both $P$ and $Q$ is TRUE
    \item $P$ is TRUE and $Q$ is FALSE
    \item $P$ is FALSE and $Q$ is TRUE
    \item both $P$ and $Q$ are FALSE
\end{enumerate}
\end{enumerate}
\end{document}
